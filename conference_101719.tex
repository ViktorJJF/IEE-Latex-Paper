\documentclass[conference]{IEEEtran}
\IEEEoverridecommandlockouts
% The preceding line is only needed to identify funding in the first footnote. If that is unneeded, please comment it out.
% \usepackage{cite}
\usepackage{amsmath,amssymb,amsfonts}
\usepackage{algorithmic}
\usepackage{graphicx}
\graphicspath{ {./images/} }
\usepackage{textcomp}
\usepackage{xcolor}

% biblatex
% \usepackage{biblatex}
% \addbibresource{bibtex/PaperDubai.bib}
% \addbibresource{zotero.bib}

\def\BibTeX{{\rm B\kern-.05em{\sc i\kern-.025em b}\kern-.08em
    T\kern-.1667em\lower.7ex\hbox{E}\kern-.125emX}}
\begin{document}

\makeatletter
\newcommand{\linebreakand}{%
    \end{@IEEEauthorhalign}
    \hfill\mbox{}\par
    \mbox{}\hfill\begin{@IEEEauthorhalign}
}
\makeatother

\title{Comparison of NLU services performance applied to frequently asked questions on a university}

\author{\IEEEauthorblockN{1\textsuperscript{st} Víctor Juan Jimenez Flores}
    \IEEEauthorblockA{\
        % textit{dept. name of organization (of Aff.)} \\
        \textit{name of organization (of Aff.)}\\
        Tacna, Perú \\
        victorjuanjf@apicyt.com}
    % asociación peruana de investigación, ciencia y tecnología comprar dominio antes de enviar paper
    \and
    \IEEEauthorblockN{2\textsuperscript{nd} Oscar Juan Jimenez Flores}
    \IEEEauthorblockA{
        % \textit{dept. name of organization (of Aff.)} \\
        \textit{Universidad Privada de Tacna}\\
        Tacna, Perú \\
        oscarjimenezflores@upt.pe}
    \and
    \IEEEauthorblockN{3\textsuperscript{rd} Juan Carlos Jimenez Flores}
    \IEEEauthorblockA{
        % \textit{dept. name of organization (of Aff.)} \\
        \textit{Southern Perú Copper Corporation}\\
        Tacna, Perú \\
        juancarlosjf@apicyt.com}
    \linebreakand % <------------- \and with a line-break
    \IEEEauthorblockN{4\textsuperscript{th} Juan Ubaldo Jimenez Castilla}
    \IEEEauthorblockA{
        % \textit{dept. name of organization (of Aff.)} \\
        \textit{Universidad José Carlos Mariátegui }\\
        Moquegua, Perú \\
        jjimenezc@ujcm.edu.pe}
}

\maketitle

\begin{abstract}
    The present paper aims to compare the main natural language understanding services in terms of performance. through F1 score and confusion matrix. To assess the performance, precision, recall and F1 score was used.
    This document is a model and instructions for \LaTeX.
\end{abstract}

\begin{IEEEkeywords}
    chatbot, Natural Language Understanding
\end{IEEEkeywords}

\section{Introduction}
Natural Language Understanding (NLU) is the ability of a machine to understand human languages. Said otherwise, it is the process of converting natural language text into a form that computers can understand \cite{pathak2017artificial}.

Throughout this paper, the term NLU

Previous work has only focused on

This paper is divided into five sections. Section II gives a brief overview of related works. The third section defines the NLU services evaluated during the research.

\section{Related works}
In recent years, multiple investigations have been carried out regarding chatbots and the impact they have on traditional processes.

Canonico and De Russis wrote a paper titled "A comparison and Critique of Natural Language Understanding Tools" \cite{Canonico2018}, which compares the main cloud-based platforms, from a descriptive and performance based point of view. Their results showed that Watsson Assistant is the platform who performs best.

In the other hand, Braun, Hernandez, Matthes and Langen wrote a paper titles "Evaluating Natural Language Understanding Services for Conversational Question Answering Systems", which presents a method to evaluate the classification performance of NLU services. Their results indicated that LUIS showed the best scores and RASA could achieve similar results.



\section{Natural language understanding services}
\subsection{Watsson Assistant}
Watsson Assistant is a cognitive cloud service offering by IBM that makes it easier for developers to build conversational interfaces and embed them into any application \cite{sabharwal2019developing}.
\subsection{Dialogflow}
Dialogflow is a Google service that runs on Google Cloud Platform. Dialogflow is a natural language understanding platform that makes it easy to design and integrate a conversational user interface into any system \cite{dialogflow2020}.
\subsection{Wit.ai}
Facebook has established the Wit.ai bot engine which allows training bots with sample conversations and have your bots repeatedly learn from interrelating with customers \cite{seligman2018artificial}.
\subsection{LUIS}
The Language Understanding Intelligent Service (LUIS) is a Microsoft's NLU cloud service and is part of the cognitive suite \cite{pathak2018iot}.
\subsection{Amazon LEX}
Amazon Lex is a development platform,belonging to Amazon, for building intelligent assistants or chatbots, which provides many AI capabilities like Automatic Speech Recognition (ASR) and Natural language Understanding (NLU) \cite{tripuraneni2019hands}.
\subsection{Rasa}
Rasa NLU is an open-source NLP library for intent classification and entity extraction in chatbots \cite{raj2018building}.

\section{Literature Review}
\subsection{Measure}
Precision
\begin{equation}
    Precision=\frac{TP}{TP+FP}\label{eq}
\end{equation}
Recall
\begin{equation}
    Precision=\frac{TP}{TP+FN}\label{eq}
\end{equation}
\begin{equation}
    F_{1} =\frac{2 \times Precision\times Recall}{Precision+Recall}\label{eq}
\end{equation}

\section{Materials and Methods}
\subsection{Materials}
The NLU services used during the research were Dialogflow, Wit.ai, LUIS, Amazon LEX and Rasa.

30 intents were used

\subsection{Procedure}
In order to identify F1 score, it was necessary to calculate the precision and recall.

The software as a service used to ---- was qbox.ai

The F1 score was chosen beacuse it is one of the most practical ways to numerically calculate the performance of an NLU service

\section{Result}
\begin{table}[htbp]
    \centering
    \caption{F1 scores}
    \begin{tabular}{|c|c|}
        \hline
        NLU Service & F1 score    \\
        \hline
        Dialogflow  & 0.820630111 \\
        \hline
        LUIS        & 0.357854738 \\
        \hline
        Watson      & 0.820979481 \\
        \hline
        Wit.ai      & 0.576004394 \\
        \hline
        LEX         & 0.503356458 \\
        \hline
        RASA        & 0.615315904 \\
        \hline
    \end{tabular}%
    \label{tab:addlabel}%
\end{table}%
\section{Discussion}
\section{Conclusion}
The limitations during the research were
\section{Future work}

\section*{Acknowledgment}

The preferred spelling of the word ``acknowledgment'' in America is without
an ``e'' after the ``g''. Avoid the stilted expression ``one of us (R. B.
G.) thanks $\ldots$''. Instead, try ``R. B. G. thanks$\ldots$''. Put sponsor
acknowledgments in the unnumbered footnote on the first page.

\bibliographystyle{IEEEtran}
% \printbibliography
\bibliography{C:/Users/JIMENEZ/Documents/bibtex/PaperIEEE.bib}

\end{document}